\documentclass[12pt]{article}
\usepackage[a4paper, hmargin={2.8cm, 2.8cm}, vmargin={2.5cm, 2.5cm}]{geometry}
\usepackage{eso-pic} % \AddToShipoutPicture

\usepackage[utf8]{inputenc}
\usepackage[T1]{fontenc}
\usepackage{lmodern}
\usepackage[english]{babel}
\usepackage{cite}
\usepackage{amssymb}
\usepackage{amsfonts}
\usepackage{amsmath}
\usepackage{enumerate}
\usepackage{mathrsfs}
\usepackage{fullpage}
\usepackage[linkcolor=red]{hyperref}
\usepackage[final]{graphicx}
\usepackage{color}
\usepackage{listings}
\renewcommand*\lstlistingname{Code Block}
\definecolor{bg}{rgb}{0.95,0.95,0.95}

%caption distinct from normal text
\usepackage[hang,small,bf]{caption}
\usepackage{hyperref}

\hypersetup{
    colorlinks,%
    citecolor=black,%
    filecolor=black,%
    linkcolor=black,%
    urlcolor=black
}

\author{
  \texttt{Mikkel Enevoldsen} \\[.4cm]
  \texttt{Kristian Høi} \\[.4cm]
  \texttt{Simon Van Beest} \\[.4cm]
  \vspace{8cm}
}

\title{
  \vspace{3cm}
  \Huge{G3} \\[.25cm]
  \large{System calls for the KUDOS filesystem}
  \vspace{.75cm}
}

\begin{document}

\AddToShipoutPicture*{\put(0,0){\includegraphics*[viewport=0 0 700 600]{includes/ku-farve}}}
\AddToShipoutPicture*{\put(0,602){\includegraphics*[viewport=0 600 700 1600]{includes/ku-farve}}}

%% Change `ku-en` to `nat-en` to use the `Faculty of Science` header
\AddToShipoutPicture*{\put(0,0){\includegraphics*{includes/ku-en}}}

\clearpage\maketitle
\thispagestyle{empty}

\newpage

%\tableofcontents %generate table of content

\thispagestyle{empty}

%\newpage
\pagestyle{plain}
\setcounter{page}{1}
\pagenumbering{arabic}

%A short report where you document your code, discuss different possibilities to solve the tasks, and explain the design decisions made (why you preferred one particular way out of several choices).

\section*{Task 1}
\subsection*{Implementation}
\textbf{Syscall\_read}\\
In syscall we read we first check if the filehandle and length is valid. If the user tries to "shoot the kernel" we throw kernel panic.\\
If the filehandle is zero we use the gcd device to read from STDIN and returns the bytes read.\\
Before calling vfs\_read with a offset of minus 2, we check if the process has opened the filehandle, if not error is returned. Lastly we call vfs\_read and return its value. \\
same goes for syscall\_write except we call vfs\_write.\\
\textbf{Syscall\_open}\\
We call vfs\_open. then we add the filehandle to the process list of open files. The filehandle returned by vfs\_open is off by 2, so we add 2. The index is saved at a free spot in the list. If list is full, error is returned. If everything went well, the filehandle is returned.\\
\textbf{Syscall\_close}\\
In sycall\_close we firstly validate the filehandle by checking if it is above 2 and the process file list contains the given filehandle. \\
Then we call vfs\_close and resets the index in the process files list.\\
\textbf{Syscall\_create}\\
In the syscall create we checks the parameters givens and call vfs\_create.
sames goes for syscall\_delete, syscall\_seek, syscall\_filecount, syscall\_file.\\

Inorder to support process files list we added a int list with size defined of PROCESS\_MAX\_FILES we made 3 helper functions in process.c.\\
Find\_free\_index() which findes a free spot in the file list.\\
Find\_index() which finds the index of a given value.\\
Close\_open\_files() which are called in process\_exit() to close all open files.

\subsection*{Testing}
We created a test file test.c, which contains calls the all syscalls.\\
\section*{Task 2}
\begin{enumerate}
  \item 4MB: BAT requires 2 blocks\\
    16 MB: 8 blocks \\
    512 MB : 256 blocks \\
    2 GB : 1024 blocks \\
  \item no, we can not fully utlize the volume, because the maximum amount of files are 25 and the maximum file size is 65024. Thus we can only use: $25 \cdot 65024\ b$ = $1.62 MB$.\\
  \item The optimal number blocks is the maximum amount of blocks for files times number    
  of  files + number of filename blocks + MD + BAT + headerblock this gives \\
  $25\cdot127 + 25 + 2 \ = 3203$
  \item If the last word of the fileheader block is a pointer then a file consists of 126 + 127 blocks. Thus the maximal filesize is : $(126+127)\cdot512 = 129.53$ Kb
  \item The double-indirect pointer adds $127^2$ new blocks. the indirect block adds another 127 blocks. The max filesize is then: $(127+127^2+125)\cdot 512b = 8.38 $ Mb
\end{enumerate}

\section*{Task 3}
1. Three jobs of length 200 with the SJF and FIFO schedulers.\\

\noindent FIFO:
$$Turnaround = 200 * 3 = 600$$ $$AvgTurnaround = (200 + 400 + 600) / 3 = 400$$\\
\noindent SJF:
$$Turnaround = 200 * 3 = 600$$     $$AvgTurnaround = (200 + 400 + 600) / 3 = 400$$\\

\noindent FIFO:
$$Response time = 0 + 200 + 400 = 600$$     $$AVG = 600/3 = 200$$\\
\noindent SJF:
$$Response time = 0 + 200 + 400 = 600$$     $$AVG = 600/3 = 200$$\\

\noindent 2. Now do the same but with jobs of different lengths: 100, 200, and 300.      (Depends on arrival in FIFO)\\

\noindent FIFO: 
$$Turnaround = 100 + 200 + 300 = 600$$     $$AvgTurnaround = (100 + 300 + 600) / 3 = 333,33     \leftarrow Best case$$
							   $$= (300 + 500 + 600) / 3 = 466,66     \leftarrow Worst case$$

\noindent SJF:
$$Turnaround = 100 + 200 + 300 = 600$$     $$AvgTurnaround = (100 + 300 + 600) / 3 = 333,33$$

\noindent FIFO: 
$$Response time = 0 + 100 + 300 = 400$$    $$Avg = 400/3 = 	133,33     \leftarrow Best case$$
      $$Response time = 0 + 300 + 500 = 800$$    $$Avg = 400/3 = 	266,66     \leftarrow Worst case$$
							   
\noindent SJF:  
$$Response time = 0 + 100 + 300 = 400$$    $$Avg = 400/3 = 	133,33$$


\noindent 3. Now do the same, but also with the RR scheduler and a time-slice of 1.\\
Response time vil være 3 så længe der er 3 jobs. Altså 0 + 1 + 2 = 3         
$$Avg = 3 / 3 = 1$$

$$Turnaround: 300 + 500 + 600 = 1400$$        $$Avg = 466,66$$

\noindent 4. For what types of workloads does SJF deliver the same turnaround times as FIFO?\\
Hvis alle jobs er af ens længde.


\section*{Task 4}

\begin{tabular}{|c|c|}
  \hline
  numpages & time\\
  \hline
  2 & 2,78\\
  4 & 3,06\\
  8 & 3,16\\
  16 & 7,12\\
  32 & 10,92\\
  64 & 12,07\\
  128 & 11,99\\
  256 & 12,42\\
  512 & 14,16\\
  1024 & 15,29\\
  \hline
\end{tabular}
Above is our timing results. The jump between 8 and 16 is due the first level cache is only 8 pages big and there for the TLB miss penalty is doubled (see. figure 19.5 in chapter 19).\\
the rest if times is maybe due timeoftheday() function? 
\end{document}
