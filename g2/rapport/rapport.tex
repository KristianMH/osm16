\documentclass[12pt]{article}
\usepackage[a4paper, hmargin={2.8cm, 2.8cm}, vmargin={2.5cm, 2.5cm}]{geometry}
\usepackage{eso-pic} % \AddToShipoutPicture

\usepackage[utf8]{inputenc}
\usepackage[T1]{fontenc}
\usepackage{lmodern}
\usepackage[english]{babel}
\usepackage{cite}
\usepackage{amssymb}
\usepackage{amsfonts}
\usepackage{amsmath}
\usepackage{enumerate}
\usepackage{mathrsfs}
\usepackage{fullpage}
\usepackage[linkcolor=red]{hyperref}
\usepackage[final]{graphicx}
\usepackage{color}
\usepackage{listings}
\renewcommand*\lstlistingname{Code Block}
\definecolor{bg}{rgb}{0.95,0.95,0.95}

%caption distinct from normal text
\usepackage[hang,small,bf]{caption}
\usepackage{hyperref}

\hypersetup{
    colorlinks,%
    citecolor=black,%
    filecolor=black,%
    linkcolor=black,%
    urlcolor=black
}

\author{
  \texttt{Mikkel Enevoldsen} \\[.4cm]
  \texttt{Kristian Høi} \\[.4cm]
  \texttt{Simon Van Beest} \\[.4cm]
  \vspace{8cm}
}

\title{
  \vspace{3cm}
  \Huge{G2} \\[.25cm]
  \large{Priority queue and syscall read/write}
  \vspace{.75cm}
}

\begin{document}

\AddToShipoutPicture*{\put(0,0){\includegraphics*[viewport=0 0 700 600]{includes/ku-farve}}}
\AddToShipoutPicture*{\put(0,602){\includegraphics*[viewport=0 600 700 1600]{includes/ku-farve}}}

%% Change `ku-en` to `nat-en` to use the `Faculty of Science` header
\AddToShipoutPicture*{\put(0,0){\includegraphics*{includes/ku-en}}}

\clearpage\maketitle
\thispagestyle{empty}

\newpage

%\tableofcontents %generate table of content

\thispagestyle{empty}

%\newpage
\pagestyle{plain}
\setcounter{page}{1}
\pagenumbering{arabic}

%A short report where you document your code, discuss different possibilities to solve the tasks, and explain the design decisions made (why you preferred one particular way out of several choices).

\section*{Documentation}

We have defined a data structure process\_control\_block\_t, containing several elements. State, pid, stack\_top, entry\_point, retval and parent. The most interesting one is state, which is of type process\_state\_t. Process\_state\_t is an enumarated type containing the variables RUNNING, FREE, ZOMBIE. The advantage of this design choice is that we won't have to update each variable. Instead we set one of them, and the rest is automatically set to 0.\\ 
The stack\_top and entry\_point is used for our spawned processes. It is urged that we consider what to do with child processes of exited processes. We would call exit on each child process recursively. We idintify wich child processes to exit, using the pointer from the child process to its parent. This should work with multiple trheads.\\\\
We introduced a helper function: find\_pid. find\_pid is straight forward. It returns the id of a process.
\\
Process\_spawn we create a new thread and setups the process on the new thread. lastly we run thread\_run on the new thread to begin execution.\\
In process\_exit we save the given return value in the process block and sets the state to zombie. 
After that we call sleepq\_wake\_all to alert other threads that called process\_join
\\
In process\_join we first acquire a pointer to the process block with the given pid. then we check if the pid has died, if not we add the resource to sleepq and call thread\_switch to sleep. When woken we return the return value from the given pid.
\subsection*{Testing}
We created a program called ptest, which spawns a new process running the program phello. phello simply prints hello and calls syscall\_exit. After ptest spawned phello it calls syscall\_join and prints the return value, in this case it prints 42. When it successfully joined syscall\_halt is called.

\end{document}
